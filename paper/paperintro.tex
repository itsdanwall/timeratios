\documentclass[]{article}

%opening
\title{An exploration of how ratios affect time discounting}
\author{Daniel Wall}


\begin{document}

\maketitle

\begin{abstract}
	Choices involving time, intertemporal choices, are ubiquitous -- for instance choosing a shipping speed for your latest Amazon order. 
	Further, people use ratios in everyday life -- intertemporal choices are no different, 
	recent cognitive models of intertemporal choice show that people calculate ratios of both money and time. 
	These models account for diminishing impatience -- when items are moved into the future, people are more patient. 
	Our goal is twofold: 1) to show that previous models of intertemporal choice do not properly account for temporal ratios and present a conceptual model of time discounting, and  2) by explicitly manipulating how people use ratios of time, we can alter how people make intertemporal choices. 
	This leads to novel predictions for how to optimally frame intertemporal choices to induce either patience or impatience. 
	Further we show that in a hypothetical shipping scenario that when temporal ratios are explicit, people are more impatient. 
	
\end{abstract}

\section{Introduction}


Many of our daily choices involve making decisions involving time in some respect, often taking the form of a the tradeoff between receiving something of smaller value now and receiving something of larger value in the future. 
For instance when a consumer selects a shipping speed when shopping online they are making an intertemporal choice.
These decisions require consumers to, perhaps implicitly, calculate how much waiting for the slower shipping speed is worth.  


Firms could have differing preferences on whether a consumer is impatient or patient when choosing a shipping speed. 
If a firm's inventory is low on a particular item they may not profit from the extra cost that it takes to expedite shipping, however if their inventory is high they may want to nudge consumers to ship their items faster. 
It is important to understand what variables consumers are using and how they are making intertemporal choices to create robust and adaptive choice environments. 
This understanding how to optimally display information leads to novel predictions for how to display shipping speeds. 

Recent models of intertemporal choice have posited that people use ratios in intertemporal choices [ITCH, DRIFT]. 
Across [] studies we show that how people make intertemporal choices is dependent on how they perceive ratios. 
Further we show that by making the ratios between times explicit we can induce a higher reliance on ratios. 
If the temporal ratio is explicit people are more impatient than when the ratio is implicit.

\section{Ratios and Intertemporal Choices}

Ratios are ubiquitous in the modern world and are paramount in disparate fields from the stock market to baking. 
When making inferences from ratios, people are biased in predictable ways.
For instance, Yamagishi showed that ratios with large numerators and denominators are perceived to be riskier than ratios with small numerators and denominators. Further, Reyna and Brainerd argued that people overweight numerators compared to denominators.
The biases used when calculating ratios are relevant to intertemporal choices. 

Recent models of intertemporal choice have posited that people use ratios to decide between a smaller sooner and a larger later option [ITCH, DRIFT]. 
An example intertemporal choice is choosing between \$5 in 1 day and \$6 in 5 days. 
The ITCH model would predict that participants calculate differences between and ratios of both the dollar amounts and the times. 
If people are biased in calculating ratios for everyday items, it follows that they would be biased in calculating ratios in intertemporal choices. 

One of the empirical regularities in intertemporal choice research is that impatience decreases when times are moved away from the present.
If we took the above example and moved both options back by 20 days we would see more patient responses. 
Multiple explanations for this phenomenon -- known as decreasing impatience -- have been marshaled. 

Zauberman showed that taking into account subjective perceptions of time can account for decreasing impatience. 
For instance they find that the subjective time perception of 1 year is not different from the subjective time perception of 3 years. 
Further subjective time perception was shown to follow a psychophysical logarithmic function. Zauberman, however, embedded this subjective time perception into a hyperbolic discounting function – an alternative based discounting function.  

Beyond accounting for diminishing impatience subjective time perception can account for subadditive discounting. 
Read (2001) found that intertemporal choices are intransitive, specifically “discounting over a delay is greater when the delay is divided into subintervals than when it is left undivided”. 
However, Zauberman showed that participants perceive the total time horizon as longer when it is divided into subintervals than when it is not divided. 
Further participants discounted more when the delay was divided into subintervals than when it was undivided. 

The focus of this paper is on how people calculate ratios specifically about time. 
We hypothesize that the ratio change in times alters the extent of decreasing impatience. 
We begin with a review of the current discounting models and how they account for decreasing impatience. 
Further we will show that while some models predict that ratio change will have an effect on decreasing impatience, that none of the models predict effects as large as we see. 
These results introduce a phenomenon whose extent is not captured by the current discounting models. 
They show that the properties of the numbers given can alter patience via their ratio. 
Moreover recent models of intertemporal choice do not properly predict the degree of decreasing impatience properly, we show that our modification to their model does a better job of predicting choices. 

We then apply our conceptual model to situations where temporal ratios are explicit. 


\section{Discounting Models}

When presented with a choice between receiving A) \$10 in 5 days or receiving B) \$12 in 7 days, there are a large number of calculations you could perform to decide which to choose.  Alternative based theories -- such as exponential and hyperbolic discounting -- posit that you calculate a discounted utility for each option. Whereas other, attribute based models, posit that people make comparisons attribute based comparisons -- they compare the money and time amounts to one another. 

\subsection{Exponential Discounting}

The normative model of intertemporal choice is exponential discounting.
The exponential discounting function is  

\begin{equation}\label{eq:exp}
	\delta_{exp} = \exp^{-rt}
\end{equation}


Where $\delta$ is the discount fraction which each monetary amount is multiplied by and $r$ is the exponential discount rate. And the utility of an option is:

\begin{equation}\label{eq:utilalt}
	U(x, t) = x \delta
\end{equation}



 If the daily exponential discount rate is .01 then the current value of A) Would be $\$10 * \exp^{-.01*5} = 9.5$ and the current value of B) would be $\$12 * \exp^{-.01*7} = 11.19$

From \cite{@Toubia2014} Toubia 2014 we can get the probability of choosing larger later from the following formula

\begin{equation}\label{eq:pll}
	P_{LL} = \frac{\exp(U_{LL})}{\exp(U_{LL})+ \exp(U_{SS})}
\end{equation}


Using this formula we get the following probability of choosing larger later $\frac{11.19}{11.19+9.5} = .54$

One of the empirical regularities in intertemporal choice research is that impatience decreases when times are moved away from the present. The exponential model, however, does not account for this phenomenon. For instance if we added 100 days to both A and B yeilding C) \$10 in 105 days and D) \$ 12 in 107 days the utility of C would be $3.5$ and the utility of D would be $4.11$. Plugging into equation \ref{eq:pll} we get $.54$, the same answer we saw before. 

\subsection{Hyperbolic}

Hyperbolic discounting has been shown to account for decreasing impatience

\begin{equation}\label{eq:hyp}
\delta_{hyp} = \frac{1}{1+kt}
\end{equation}

Where $k$ is the hyperbolic discount rate, and the utility of the option is found by equation \ref{eq:utilalt}.

Assuming daily $k = .01$, for choice A) \ref{eq:hyp} we get $\delta_A = \frac{1}{1+.01*5} = .952$ meaning the utility is $9.52$ and for B) the utility is $11.21$ which yields a larger later probability of $.541$. When we look at the larger later probabilities for C and D we get, 4.88 and  5.79 yielding a larger later value of $.543$. There's an increase in the likelihood of choosing the larger later option when both options are moved into the future, this is known as diminishing impatience. 

The hyperbolic and exponential models both predict that people make alternative based transitions; however recent models have shown that people make attribute based transitions



\subsubsection{Subjective Time Perception}

Zauberman showed that peoples subjective perception of time can account for hyperbolic discounting. 

\subsection{ITCH}
\label{ITCH}

Marzilli Ericson and colleagues showed that fully attribute based models can predict out of sample choices better than alternative based models. 
According to ITCH the probability of choosing larger later is

\begin{equation}\label{eq:itch}
	P(LL) = L \left(\beta_I + \beta_{xA}(x_2 - x_1) + \beta_{xR} \frac{x_2 - x_1}{x^*} + \beta_{tA}(t_2 - t_1) + \beta_{tR} \frac{t_2 - t_1}{t^*}\right)
\end{equation}

In the paper they give the example parameters for each of the coefficients of 0, .1, .1. -.1, -.1. Using these coefficients in equation \ref{eq:itch} for A compared to B the probability of choosing larger later would be $.496$, whereas for C and D the probability of choosing larger later would be $.504$


\subsection{Discounting By Intervals}

\begin{equation}\label{eq:int}
	D(t_S, t_L) = \left(1 + \alpha \left(\frac{w(t_L) - w(t_S)}{\vartheta}\right)^{\vartheta}\right)^{-\beta/\alpha}
\end{equation}


\subsection{Trade-off model}

% %\begin{equation}
% %Q(w(t_L) - w(t_S)) = 
% %\begin{cases}
% %v(x_L) - v(x_S), & \text{if}\ x_L > x_S > 0 \\
% %v(x_S) - v(x_L), & \text{if}\ 0 > x_S > x_L
% %\end{cases}
% %\end{equation}

\subsection{Summary}
Taken together these models show that some models predict that a large ratio change will have more decreasing impatience than a small ratio change; however we posit that the degree of  decreasing impatience will be larger when the ratio change is small compared to when it is large.
Take the following two choices for example $SS_{sr}$ = \$10 in 5 days and $LL_{sr}$ = \$12 in 6 days and $SS_{lr}$ = \$10 in 5 days and $LL_{lr}$ = \$12 in 10 days. Where $SS_{sr}$ stands for the smaller sooner option for the small time ratio. 
If we add 5 days to both items they would now be $SS_{sr}^{del}$ = \$10 in 10 days and $LL_{sr}^{del}$ = \$12 in 11 days and $SS_{lr}^{del}$ = \$10 in 10 days and $LL_{lr}^{del}$ = \$12 in 15 days. 

[Insert a table about the degree of decreasing impatience for each model here]

\section{Time Ratio Model (TRaM)}

We propose a new model of ITC which can account for the degree of decreasing impatience based on the ratios of time. 
We begin with a verbal description of our model

\subsection{An example of TRaM}

When presented with a choice between \$5 in 1 day and \$10 in 5 days. 
A TRM decision maker would look at the two delays and calcluate their ratio. 
Then they will use a rule of thumb to say how much relative compensation they need to account for that ratio.
Below is an example of a TRM decision maker would make an intertemporal choice:

The time is 5 times as long therefore I will need about 3 times as much money. 

If people use this heuristic then the ratio between times becomes important. 
If we move both options up by five days then the ratio between times becomes $10/6$ = 1.66. A participant using the TRM heuristic would say that 
The larger later time is about 1.5 times as long as the smaller sooner time therefore I need about 1.25 times as much money. 

Further TRaM predicts that the salience of the temporal ratio means that it is more likely to be used. 
For instance if the temporal ratio explicit, participants reliance on it would increase.



\section{Combining ITCH, DRIFT, and TRaM }

Read, Frederick, and Scholten (2013) show that when making intertemporal choices how the monetary amounts are framed alters patience. 
They propose a Difference, Ratio, Interest, Finance, and Time (DRIFT) model of intertemporal choice. 
For instance when an intertemporal choice is framed as \$100 now OR \$110 in 1 year people will rely on differences in monetary amounts.
However if the choice is framed as \$100 now or an extra 10 \% after 1 year, people will rely more on ratios. 
The DRIFT model of intertemporal choice shows that how monetary amounts are framed can alter levels of patience.

Relatedly Read, Frederick, Orsel, and Rahman showed that how the times intertemporal choice are displayed can affect patience. Specifically Read and colleagues found that when times are described as dates (e.g. December 30th) people are more patient than when described as delays (e.g. in 1 month). Temporal framing can alter patience. 

Given that temporal framing can alter patience and that people rely on ratiosin intertemporal choice, we propose that making ratios between times explicit can alter patience. 
For instance, an intertemporal choice between receiving \$10 in 2 days and \$11 in 12 days could be re-framed as \$10 in 2 days and \$11 in 6 times as long. 
We proposed that this temporal framing leads to a greater weight being placed upon the time ratio term $\beta_{tR}$ and a lower weight on the time difference term $\beta_{tA}$ in equation \ref{eq:itch} . Taking the example and using the parameters in section \ref{ITCH} the probability of choosing larger later is $.26$. However if the ratio is weighted more heavily -- the absolute value of the beta coefficient goes up e.g.  $\beta_{tR} = -.5 $ -- and the difference is weighted the same, eg. $\beta_{tA} = -.1$, the probability of choosing larger later is now $.0065$. 

Holding everything else constant by making the temporal ratio more salient, e.g. increasing the absolute value of $\beta_{tR}$,  monotonically decreases the probability of choosing larger later. We propose that making temporal ratios more salient increases their decision weight, leading to more impatient choices

{\large [This CANNOT account for the change in the decision weight on temporal differences by making temporal ratios salient, which I am unsure of how to work out]. 
}

The TRaM conceptual model of intertemporal choice suggests that by making temporal ratios explicit that people's reliance on them will be higher.
When people's reliance on temporal ratios is higher, the weight in which temporal ratios receive will be higher. 
This weight has predictable consequences on how impatient participants are, which suggests ways in which firms can alter patience based on how times are presented. 




\section{Study 1}

In study 1 we will test this prediction of the degree of decreasing impatience depending on the change in ratio.
We predict that when ratio changes are large, participants will show more diminishing impatience than when ratio changes are small. 

\subsection{Methods}
Participants were in one of two conditions. 
Six intertemporal choices were consistent across conditions and were used to calculate the participants discount fraction.
The other two intertemporal choices varied across conditions, these choices varied in the extent to which the temporal ratio changed.
We varied the ratio with either a small or large time ratio which was moved into the future by 5 days. 

In condition 1 participants the two focal questions were:  
\$10 in 5 days or \$11 in 6 days
and 
\$10 in 10 days or \$11 in 11 days

In condition 2 participants the two focal questions were: 
\$10 in 5 days or \$11 in 15 days
and 
\$10 in 10 days or \$11 in 20 days

In condition 1 the change in temporal ratio is small (1.2 to 1.1), whereas in condition 2 the change in temporal ratio is large (3 to 2). 


\subsubsection{Participants}

We recruited [] Rutgers university undergraduates to complete the study for course credit. 

\subsection{Results}
 
We calculated the daily discount fraction for each of the four focal items while controlling for the participants discount fraction from the other 6 items. 
The degree of decreasing impatience was calculated by taking the difference between the daily discount fraction from the two focal items. 
We ran a multilevel linear regression predicting the degree of decreasing impatience with condition and discount fraction of the other items as predictors. 

The regression showed that the degree of decreasing impatience was predicted by condition.  

\subsection{Discussion}

Study 1 shows that the use of temporal ratios is more important than other cognitive models predicted. 
If temporal ratios are important, by explicitly manipulating them, we should be able to alter levels of patience

\section{Study 2}

In study 2 we manipulate how easy it is to compare temporal ratios. 
From ITCH and TRaM we hypothesize that participants who are shown explicit temporal ratios will make more impatient decisions 

\subsection{Methods}

Each respondent answered 27 questions framed as intertemporal choices.
We varied the time to the smaller sooner option ($SS_{time} =$ 5, 10, or 50 days), 
the ratio between the smaller sooner and larger later times ($SS_{time} / LL_{time}  = $ 1.2, 2.2, or 3.2 times as long), 
 the difference between the smaller sooner and larger later amount ($LL_{amnt} - SS_{amnt} =$ \$5, \$10, \$30). All smaller sooner amounts were \$47 [Or we could pull from a uniform distribution?].

Each respondent was assigned to one of two frames: implicit temporal ratios and explicit temporal ratios. An example of both is given below

\textbf{Implicit Temporal Ratio: } You can receive \$47 in 5 days or \$54 in 6 days

\textbf{Explicit Temporal Ratio: } You can receive \$47 in 5 days or \$54 in 1.2 times as long

\subsection{Results}

We perform a multilevel logistic regression predicting choice with varying intercepts for each participant, the three within subject factors, and the between subject factor of temporal ratio frame. 

The results of the logistic regression are shown in table []. We see that participants in the explicit temporal ratio are more impatient frame than those in the implicit temporal ratio frame. [Is there an interesting interaction here e.g. we see this effect only when the ratio is large but not when it is small?]


\section{Study 3}

While study 2 showed that when time ratios are explicit, people are more impatient, it was framed as gains not losses.
Most shipping situations involve a cost to the consumer. 
In Study 3 we replicate study 2 but with losses.

\subsection{Methods}

The methods for study 3 are identical to study 2 except that all monetary amounts are framed as losses and not gains. 

\subsection{Results}

We replicate the findings of study 2 when framed as losses. 

\section{Study 4}

Studies 1 and 2 showed that explicit time ratios make people more impatient for both gains and losses, but do not specifically deal with a shipping scenario. Study 4 uses a hypothetical shipping scenario.

\subsection{Methods}

Participants were told to imagine that they had just decided to buy \$75 worth of merchandise on a popular on-line vendor. 
[Should this strictly be between subjects?]

They then were presented with two shipping options in one of the two frames.

\textbf{Implicit Temporal Ratio}

Option 1: Receive your order in two days for \$7

Option 2: Receive your order in 10 days for \$3
 
 
\textbf{Explicit Temporal Ratio}

Option 1: Receive your order in two days for \$7

Option 2: Receive your order in 5 times longer than option 1 for \$3

\subsection{Results}

We perform a logistic regression with shipping choice as the dependent measure and explicit vs implicit temporal ratio as the independent variable. 
We find that participants are more likely to delay their shipping in the implicit temporal ratio condition. 

\subsection{Discussion}

Study 4 shows that temporal ratios are an important driver of real world shipping choices. 
By making temporal ratios explicit, people are more likely to be patient. 

\section{General Discussion}

Across four studies we show temporal ratios are an important driver of temporal choice. 
This importance has been understated by previous models of intertemporal choice, and by explicitly expressing temporal ratios people are more impatient. 

\end{document}
